\documentclass{article}

% Packages for mathematics and special characters
\usepackage[utf8]{inputenc}
\usepackage[T1]{fontenc}
\usepackage{amsmath}
\usepackage{amsfonts}
\usepackage{amssymb}
\usepackage[english]{babel}
\usepackage{tcolorbox} % Add this package

\title{Experiment}
\author{Noirin Graham}
\date{\today}

% Define the boxes
\newtcolorbox{problembox}{colback=blue!5!white, colframe=blue!75!black, sharp corners=southwest, boxrule=0.8mm, title=Problem}
\newtcolorbox{solutionbox}{colback=green!5!white, colframe=green!75!black, sharp corners=southwest, boxrule=0.8mm, title=Solution}

\begin{document}

\maketitle

\begin{problembox}
The wheels of the Robot should movable 360° around the x axes. For that a continuous type needs to be chosen. We need to find out which axes can move and which can't. Also, the connection to the body is different.
\end{problembox}

\begin{solutionbox}
% Add your solution here
This is where you will provide the solution to the problem.
\end{solutionbox}

\end{document}
