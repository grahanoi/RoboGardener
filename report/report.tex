\documentclass{article}

% Packages for mathematics and special characters
\usepackage[utf8]{inputenc}
\usepackage[T1]{fontenc}
\usepackage{amsmath}
\usepackage{amsfonts}
\usepackage{amssymb}
\usepackage[english]{babel}
\usepackage{tcolorbox}

\title{Experiments in Robot Modeling with ROS}
\author{Noirin Graham}
\date{\today}

% Define the boxes
\newtcolorbox{problembox}{colback=blue!5!white, colframe=blue!75!black, sharp corners=southwest, boxrule=0.8mm, title=Problem}
\newtcolorbox{solutionbox}{colback=green!5!white, colframe=green!75!black, sharp corners=southwest, boxrule=0.8mm, title=Solution}

\begin{document}

\maketitle

\section*{Experiment 1: Sourcing the ROS Environment}

\begin{problembox}
\textbf{Problem:} In a Linux environment with ROS, each new shell requires sourcing the ROS environment using the command: \texttt{source install/setup.bash}. This must be repeated every time a new terminal is opened, which is inefficient and prone to errors.
\end{problembox}

\begin{solutionbox}
\textbf{Solution:} To streamline the workflow, the environment can be automatically sourced by adding the following command to the shell configuration file:

\begin{verbatim}
echo "source /opt/ros/rolling/setup.bash" >> ~/.bashrc
\end{verbatim}

However, if this command does not work as expected, it may result in the error message:

\begin{verbatim}
bash: /home/noi/pw2/install/setup.bash: No such file or directory
\end{verbatim}

\textbf{Steps to Resolve:}
\begin{enumerate}
    \item Open the configuration file with a text editor using the command:
    \begin{verbatim}
    nano ~/.bashrc
    \end{verbatim}
    \item Scroll to the end of the file and review the lines that source the ROS environment. Ensure the correct ROS distribution is specified:
    \begin{verbatim}
    source /opt/ros/rolling/setup.bash
    source ~/pw2/install/setup.bash
    \end{verbatim}
    \item Verify the file path is accurate and points to the correct setup.bash file corresponding to your ROS distribution. Adjust if necessary.
\end{enumerate}

\textbf{Rationale:} This approach automates environment setup, reducing manual steps and minimizing the potential for errors related to incorrect sourcing.
\end{solutionbox}

\end{document}
