\chapter{Introduction}

\section{Background}
Within the scope of this research project, a digital twin should be created with the help of the Robot Operating System (ROS2). ROS2 is an open-source software framework that defines the components, interfaces, and tools for building advanced robots. 

A robot typically consists of three primary elements:
\begin{itemize}
    \item Actuators, which enable the robot to perform movements.
    \item Sensors, which allow the robot to perceive its environment.
    \item A control system, which functions as the robot's brain, processing inputs and coordinating actions.
\end{itemize}

ROS2 enables users to build these components and create connections between them using ROS2 tools, known as topics and messages. Simulation environments such as Rviz and Gazebo provide powerful tools for developing and testing robotic systems in a virtual space before deploying them in the real world.

\section{Robot Description}
The robot to be simulated is custom-built by Zürcher Fachhochschule für Angewandte Wissenschaften (ZHAW). It features a compact, table-like design constructed out of 3D-printed profiles with dimensions of 20mm by 20mm by 500mm. The robot is equipped with four wheels, two of which are independently controllable. The back wheels are powered by individual motors, enabling differential steering for precise maneuvering. 

The control system is distributed across a Raspberry Pi and an Arduino unit. The envisioned application for this robot is in an agricultural context, specifically as an autonomous gardener operating on rooftops with solar panels. Its task would be to identify and remove tall, harmful weeds that could cast shadows on the solar panels, thereby supporting sustainable urban environments.

\section{Farmbot Integration}
As a starting point for controlling the robot, the open-source Farmbot software was utilized. Farmbot is an open-source CNC (Computer Numerical Control) farming project that provides a Cartesian coordinate farming robot and corresponding software. The Farmbot system simplifies robot control through its intuitive interface, although it has limitations in terms of operational area and movement, as the robot is restricted to predefined rails.

\section{Project Objectives}
The objectives of this project are as follows:
\begin{enumerate}
    \item \textbf{URDF File Development}: Create a Unified Robot Description Format (URDF) file that captures the robot's physical and functional characteristics, including its kinematics, dynamics, and sensor integration.
    \item \textbf{Rviz Visualization}: Visualize the robot’s URDF model in a 3D environment using Rviz.
    \item \textbf{Gazebo Simulation}: Simulate the robot within the Gazebo environment to test its autonomous functions, sensor interactions, and control systems.
    \item \textbf{Digital Twin Integration}: Establish a cohesive digital twin by integrating the simulated robot with the physical robot, enabling real-time synchronization.
    \item \textbf{Edge and Centralized Computing Implementation}: Implement edge computing with Raspberry Pi units for real-time data processing and use centralized computing for more complex tasks.
\end{enumerate}

\section{Structure of the Report}
This report is organized as follows:
\begin{itemize}
    \item \textbf{Chapter 1}: Introduces the background, robot description, and project objectives.
    \item \textbf{Chapter 2}: Reviews the relevant literature for the project.
    \item \textbf{Chapter 3}: Explains the methodology, including setting up the virtual machine, creating the URDF file, and simulating with Rviz and Gazebo.
    \item \textbf{Chapter 4}: Presents the results of the simulations and analyzes them.
    \item \textbf{Chapter 5}: Discusses the conclusions and future work based on the findings.
\end{itemize}

